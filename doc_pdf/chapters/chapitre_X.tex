\chapter{Bilan et retour d'expérience (REX)}

\section{Objectifs atteints et non atteints}

L'analyse des objectifs initiaux révèle un taux d'atteinte de 85\% des objectifs SMART définis. Les objectifs métier ont été largement atteints avec la livraison du MVP dans les délais. Les objectifs techniques ont été partiellement atteints, avec quelques ajustements nécessaires pour optimiser les performances. Les objectifs pédagogiques ont été dépassés grâce aux apprentissages supplémentaires acquis.

Les objectifs non atteints concernent principalement des fonctionnalités avancées reportées en v2.0 pour respecter les contraintes temporelles. Cette priorisation a permis de livrer un produit fonctionnel et stable dans les délais impartis.

\begin{exemple}
\textbf{Bilan des objectifs SMART :}
\begin{center}
\begin{tabular}{|l|l|l|l|}
\hline
\textbf{Objectif} & \textbf{Statut} & \textbf{Mesure} & \textbf{Commentaire} \\
\hline
Réduction temps reporting & \mycheckmark Atteint & -42\% & Dépassé l'objectif de -40\% \\
Livraison MVP 6 mois & \mycheckmark Atteint & 5.5 mois & Livré en avance \\
Adoption utilisateurs & \mywarning Partiel & 78\% & Objectif 90\%, formation nécessaire \\
Performance P95 < 500ms & \mycheckmark Atteint & 320ms & Dépassé l'objectif \\
Sécurité 0 vulnérabilité & \mycheckmark Atteint & 0 & Objectif atteint \\
\hline
\end{tabular}
\end{center}

\textbf{Objectifs non atteints :}
\begin{itemize}
    \item \textbf{Analytics avancées :} Reporté en v2.0 (complexité technique)
    \item \textbf{Intégrations externes :} Reporté en v2.0 (priorités métier)
    \item \textbf{Mobile native :} Reporté en v2.0 (PWA suffisant)
    \item \textbf{IA prédictive :} Reporté en v2.0 (ROI incertain)
\end{itemize}
\end{exemple}

\begin{conseil}
\begin{itemize}
    \item Analyser objectivement l'atteinte des objectifs
    \item Identifier les causes des non-atteintes
    \item Documenter les ajustements nécessaires
    \item Prévoir les actions correctives pour v2.0
    \item Communiquer les résultats aux parties prenantes
\end{itemize}
\end{conseil}

\begin{jury}
\begin{itemize}
    \item Quels objectifs avez-vous atteints ?
    \item Pourquoi certains objectifs n'ont-ils pas été atteints ?
    \item Comment mesurez-vous le succès de votre projet ?
    \item Avez-vous ajusté vos objectifs en cours de projet ?
    \item Quels sont vos objectifs pour la v2.0 ?
\end{itemize}
\end{jury}

\section{Difficultés rencontrées et solutions}

Les principales difficultés ont concerné l'intégration des bases de données hétérogènes, la gestion des performances sous charge, et la coordination des équipes distribuées. Chaque difficulté a été analysée pour identifier les causes racines et implémenter des solutions durables.

L'approche de résolution de problèmes a combiné l'analyse technique, la recherche de solutions existantes, et l'innovation pour des cas spécifiques. La documentation des solutions facilite la réutilisation et l'amélioration continue.

\begin{exemple}
\textbf{Tableau risques \myarrow mitigation \myarrow résultat :}
\begin{center}
\begin{tabular}{|l|l|l|l|}
\hline
\textbf{Risque} & \textbf{Mitigation} & \textbf{Résultat} & \textbf{Apprentissage} \\
\hline
Performance DB & Index + cache Redis & Latence -60\% & Cache stratégique \\
Intégration équipes & Daily standups & Communication +40\% & Processus agile \\
Sécurité données & Chiffrement + audit & 0 incident & Sécurité by design \\
Délais serrés & MVP + priorités & Livraison à temps & Focus sur l'essentiel \\
Complexité technique & Architecture simple & Maintenance facile & KISS principle \\
\hline
\end{tabular}
\end{center}

\textbf{Exemple de difficulté résolue :}
\begin{verbatim}
Problème: Latence élevée des requêtes PostgreSQL
+-- Symptômes
|   +-- Temps de réponse > 2s
|   +-- Timeout des requêtes complexes
|   +-- Surcharge CPU base de données
+-- Analyse
|   +-- Requêtes sans index appropriés
|   +-- Jointures sur de gros volumes
|   +-- Pas de cache applicatif
+-- Solutions implémentées
|   +-- Création d'index composites
|   +-- Optimisation des requêtes
|   +-- Mise en place de Redis cache
|   +-- Pagination des résultats
+-- Résultat
    +-- Latence réduite à 200ms
    +-- CPU base stabilisé
    +-- Expérience utilisateur améliorée
\end{verbatim}
\end{exemple}

\begin{conseil}
\begin{itemize}
    \item Documenter toutes les difficultés rencontrées
    \item Analyser les causes racines des problèmes
    \item Rechercher des solutions existantes avant d'innover
    \item Tester les solutions avant déploiement
    \item Partager les apprentissages avec l'équipe
\end{itemize}
\end{conseil}

\begin{jury}
\begin{itemize}
    \item Quelles ont été vos principales difficultés ?
    \item Comment avez-vous résolu ces difficultés ?
    \item Avez-vous documenté vos solutions ?
    \item Ces difficultés étaient-elles prévisibles ?
    \item Comment éviterez-vous ces difficultés à l'avenir ?
\end{itemize}
\end{jury}

\section{Dettes techniques et apprentissages}

Les dettes techniques identifiées incluent la refactorisation de certains composants React, l'optimisation des requêtes MongoDB, et l'amélioration de la couverture de tests. Ces dettes sont documentées avec des priorités et des estimations pour faciliter la planification des futures itérations.

Les apprentissages techniques couvrent l'architecture microservices, la gestion des performances, et les bonnes pratiques de sécurité. Ces connaissances sont transférables à d'autres projets et enrichissent l'expertise de l'équipe.

\begin{exemple}
\textbf{Registre des dettes techniques :}
\begin{center}
\begin{tabular}{|l|l|l|l|l|}
\hline
\textbf{Dette} & \textbf{Priorité} & \textbf{Effort} & \textbf{Impact} & \textbf{Planification} \\
\hline
Refactor composants React & Moyenne & 2 semaines & Maintenabilité & v1.2 \\
Optimisation requêtes Mongo & Haute & 1 semaine & Performance & v1.1 \\
Tests E2E manquants & Haute & 1 semaine & Qualité & v1.1 \\
Documentation API & Basse & 3 jours & Développement & v1.3 \\
Migration TypeScript & Moyenne & 3 semaines & Robustesse & v2.0 \\
\hline
\end{tabular}
\end{center}

\textbf{Apprentissages transférables :}
\begin{itemize}
    \item \textbf{Architecture :} Pattern Repository pour l'abstraction des données
    \item \textbf{Performance :} Stratégies de cache multi-niveaux
    \item \textbf{Sécurité :} Implémentation JWT avec refresh tokens
    \item \textbf{Tests :} Pyramide de tests avec couverture optimale
    \item \textbf{DevOps :} Pipeline CI/CD avec déploiement blue-green
\end{itemize}

\textbf{Exemple d'apprentissage concret :}
\begin{lstlisting}[language=JavaScript]
// AVANT : Gestion d'état complexe
const [projects, setProjects] = useState([]);
const [loading, setLoading] = useState(false);
const [error, setError] = useState(null);

// APRÈS : Hook personnalisé réutilisable
const useProjects = () => {
  const [state, setState] = useState({
    data: [],
    loading: false,
    error: null
  });

  const fetchProjects = useCallback(async () => {
    setState(prev => ({ ...prev, loading: true }));
    try {
      const projects = await projectService.getAll();
      setState({ data: projects, loading: false, error: null });
    } catch (err) {
      setState(prev => ({ ...prev, loading: false, error: err.message }));
    }
  }, []);

  return { ...state, fetchProjects };
};
\end{lstlisting}
\end{exemple}

\begin{conseil}
\begin{itemize}
    \item Identifier et documenter toutes les dettes techniques
    \item Prioriser les dettes selon leur impact et urgence
    \item Planifier la résolution des dettes dans les futures versions
    \item Capitaliser sur les apprentissages pour les futurs projets
    \item Partager les bonnes pratiques avec l'équipe
\end{itemize}
\end{conseil}

\begin{jury}
\begin{itemize}
    \item Quelles dettes techniques avez-vous identifiées ?
    \item Comment priorisez-vous ces dettes ?
    \item Quels apprentissages tirez-vous de ce projet ?
    \item Ces apprentissages sont-ils transférables ?
    \item Comment capitalisez-vous sur ces expériences ?
\end{itemize}
\end{jury}

\section{Liens utiles}

\begin{itemize}
    \item Postmortems (Google SRE): \url{https://sre.google/sre-book/postmortem-culture/}
    \item Technical Debt: \url{https://martinfowler.com/bliki/TechnicalDebt.html}
    \item Retrospectives: \url{https://www.atlassian.com/team-playbook/plays/retrospective}
    \item Lessons Learned: \url{https://bit.ly/lessons-learned}
    \item Knowledge Management: \url{https://bit.ly/knowledge-management}
\end{itemize}
