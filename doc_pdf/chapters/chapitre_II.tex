\chapter{Cadrage et cahier des charges}

\section{Objectifs métier, techniques et pédagogiques}

Dans cette section, vous devez définir clairement les trois types d'objectifs de votre projet. Le jury attend une distinction nette entre les objectifs métier (bénéfices pour l'entreprise), techniques (performance, architecture) et pédagogiques (apprentissages CDA).

\begin{for_you}{Objectifs métier :}
    \textit{
    \newline
    1. Controler les acces / heur d'arriver.
    \newline
    2. Reduciton materiel.
    \newline
    3. Securiser le batiment.
    \newline
    4. Fludifier le trafique.
    }
\end{for_you}


\textbf{Objectifs techniques :} \textit{
    \newline
    1. Crée une application de controle d'acces et integrée un systeme de badge sur telephone.
    \newline
    2. Clef de chiffrement, autentification (Identifient/Mot de passe).
    \newline
    3. Crée une interface intuitive et minimaliste.
    }

\textbf{Objectifs pédagogiques :} \textit{
    \underline{C'est quoi ??}
    \newline
}

\begin{for_you}{Tableau MoSCoW et Diagramme de périmètre MVP :}

    \textbf{Tableau MoSCoW : \newline}
    \begin{center}
        \begin{tabular}{|l|l|l|}
            \hline
                \textbf{Priorité} & \textbf{Fonctionnalité} & \textbf{Justification} \\
            \hline
                Must Have & Application & Application inutile sans \\
                Must Have & Authentification et chiffrement & Sécurité obligatoire \\ %prioriter
                Should Have & Gen badge securiser et logique & Securiser le batiment\\ %prioriter moins importante
                Could Have & Relier L'API de l'intra aux read badger & Accéder à la db\\ %bonnus pas obligatoir mais amelioration
                Won't Have & Integrée le read badgeur dans l'intra & Question pratiques \\ %Pour une V2
            \hline

        \end{tabular}
    \end{center}

    \textbf{\newline Diagramme de périmètre MVP :}
        \begin{center}
            \includegraphics[width=15cm,height=80mm]{./assets/diagramme_de_perimetre.png}
        \end{center}
\end{for_you}

\begin{conseil}
\textbf{Ce que le jury attend dans cette section :}
\begin{itemize}
    \item Une distinction claire entre objectifs métier, techniques et pédagogiques
    \item Une priorisation MoSCoW justifiée et documentée
    \item Un périmètre MVP bien délimité et réaliste
    \item Des critères d'acceptation mesurables et vérifiables
    \item Une analyse des contraintes et risques identifiés
\end{itemize}

\textbf{Conseils de rédaction :}
\begin{itemize}
    \item Quantifiez vos objectifs (pourcentages, délais, volumes)
    \item Justifiez chaque priorité MoSCoW par des arguments métier
    \item Montrez la cohérence entre MVP et objectifs SMART
    \item Utilisez des diagrammes pour visualiser le périmètre
\end{itemize}
\end{conseil}

\begin{jury}
\textbf{Questions de contrôle du jury :}
\begin{itemize}
    \item Pouvez-vous distinguer clairement vos objectifs métier, techniques et pédagogiques ?
    \item Comment avez-vous priorisé vos fonctionnalités avec la méthode MoSCoW ?
    \item Votre MVP est-il vraiment minimal et réaliste ?
    \item Quels sont vos critères de succès mesurables ?
    \item Comment gérez-vous les changements de périmètre ?
    \item Avez-vous identifié les contraintes et risques du projet ?
\end{itemize}
\end{jury}

\section{Cibles et parties prenantes}

Dans cette section, vous devez identifier et analyser vos utilisateurs cibles et toutes les parties prenantes du projet. Le jury attend une compréhension claire des besoins de chaque groupe et de leur influence sur le projet.
\newline
\begin{for_you}{Personae utilisateurs :}
    
    \textit{\newline
    1. \underline{Etudiants}: Accede aux batiment de 8h00 a 17h00 du lundi aux vendredi. \newline
    2. \underline{Staff}: Accede aux batiment de 8h00 a 19h00 du lundi aux samedi. \newline
    3. \underline{Admin}: N'a aucune restrictions.
    }
\end{for_you}

\textit{\newline}

\begin{for_you}{Parties prenantes :}

\textit{\newline
1. \underline{Etudiants de colint}: Il on besoins d'acceder aux batiment durrent sont année scolaire. \newline
2. \underline{Inviter pour colint}: Badge crée sur demande en fonctrion du besoins. Par exemple un parents n'aura acces que quelque heur aux batiment alors qu'un intervenant peux avoir un badge d'une semaine.\newline
3. \underline{Staff}: Il on besoins d'acceder aux batiment durrent leur jours de travail.
}
\end{for_you}

\textit{\newline}

\begin{for_you}{Matrice d'influence :}

\textit{Pour chaqu'une de ses partie il on un interet comment fludifier leur
trafique aux sein de l'ecole. Mais egalement securiser le batiment.}
\end{for_you}

La cartographie des parties prenantes facilite la communication et la gestion des attentes tout au long du projet. Cette approche systémique garantit que tous les besoins sont pris en compte dans la conception de la solution.

\begin{for_you}{User Stories prioritaires et Critères d'acceptation :}

\textbf{User Stories prioritaires :}
\begin{itemize}
    \item En tant que \textbf{Etudiants/Coatchs}, je veux acceder aux batiment pour venir travail
    \item En tant que \textbf{Iniviter}, je veux acceder aux batiment.
    \item En tant que \textbf{Admin}, je veux acceder aux batiment et acceder a une base de donée des Entrée/Sortie et gerer les rôles.
    \newline
\end{itemize}

\textbf{Critères d'acceptation :}
    \begin{itemize}
        \item En tant Etudiants/Coatchs/Inviter ou bien admin Je souhaite acceder aux batimentet, j'ai mon badge avec moi
        je scane ce dernier lorsque je scane le systeme verifie que j'ai un badge valide en verifient depuis la base de donée mon badge. Mon badge est reconnus
        alors la porte souvre.
        \item En tant qu'administrateurs, Je souhaite acceder a mon application de controle d'acces etant donnée que je suis déconnecté et que je suis sur la page de connexion,
        lorsque je remplis les champs «identifiant» et «Mot de passe» avec mes informations d'authentification et que je clique sur le bouton Connexion,
        alors le système me connecte
    \end{itemize}
\end{for_you}

\begin{conseil}
    \begin{itemize}
        \item Créer des personae détaillés avec leurs besoins spécifiques
        \item Identifier toutes les parties prenantes du projet
        \item Analyser l'influence et l'intérêt de chaque partie prenante
        \item Définir des user stories avec critères d'acceptation clairs
        \item Organiser des sessions de validation avec les utilisateurs
    \end{itemize}
\end{conseil}

\begin{jury}
    \begin{itemize}
        \item Qui sont vos utilisateurs cibles principaux ?
        \item Comment avez-vous validé vos user stories ?
        \item Quelles parties prenantes ont le plus d'influence ?
        \item Vos critères d'acceptation sont-ils mesurables ?
        \item Comment gérez-vous les besoins contradictoires ?
    \end{itemize}
\end{jury}

\section{Exigences fonctionnelles}

Les exigences fonctionnelles définissent précisément ce que le système doit faire pour répondre aux besoins métier. Elles couvrent les fonctionnalités front-end (interface utilisateur), back-end (logique métier), la gestion des rôles et droits d'accès, ainsi que les aspects de confidentialité et d'authentification. Cette spécification détaillée guide le développement et sert de référence pour les tests d'acceptation.

La sécurité et la confidentialité des données constituent des exigences critiques qui influencent directement l'architecture technique et les choix de développement. L'authentification robuste et la gestion fine des autorisations sont essentielles pour protéger les informations sensibles.

\subsection{Fonctionnalités « Front Office »}

Dans cette sous-section, vous devez détailler toutes les fonctionnalités accessibles aux utilisateurs finaux. Le jury attend une description précise de l'interface utilisateur et des interactions possibles.

\textbf{Fonctionnalités Front Office :} \textit{Pour les Inviter, Etudiants, Coatchs il ne pourrons acceder qu'a leur badge qui sera soit sous la forme physique ou virtuel depuis leur telephone. Quand aux administrateurs ils aurons acces en plus a une partis pour gerer et crée les badge avec des 
donée sur qui Rentre/Sorte de l'etablisement.}

\begin{for_you}{Fonctionnalités Front Office :}

    \begin{itemize}
        \item \textbf{Information personnelle :} Information pour crée un badge, sous la forme d'un formulaire (nom, prenon, email, id...).
        \item \textbf{Badge :} Acces a aux batiment.
    \end{itemize}
\end{for_you}

\subsection{Fonctionnalités « Back Office »}

Dans cette sous-section, vous devez présenter les fonctionnalités administratives et de gestion du système. Le jury attend une distinction claire entre les fonctions métier et les fonctions d'administration.

\begin{for_you}{Fonctionnalités Back Office :}
    \begin{itemize}
        \item \textbf{Gestion de badge :} Création, modification, désactivation badge.
        \item \textbf{Gestion rôles :} Attribution et modification des permissions.
        \item \textbf{Gestion de mail :} Recevoir et ajouter sont badge a sont wallet.
        \item \textbf{Rapports système :} Logs, métriques, statistiques d'usage.
        \item \textbf{Controle entrée :} Trace écrite des Entrée/Sortie.
    \end{itemize}
\end{for_you}

\subsection{L'utilisateur (public)}

Dans cette sous-section, vous devez définir clairement qui peut accéder au système et dans quelles conditions. Le jury attend une analyse des différents types d'utilisateurs et de leurs besoins spécifiques.

\begin{for_you}{Types d’utilisateurs :}

    \begin{itemize}
        \item \textbf{Etudiants et coatchs :} Acces a sont badge.
        \item \textbf{Invités :} Acces a un badge temporaire
        \item \textbf{Administrateurs :} Accès a l'application de Statistiques mais également a sont badge.
    \end{itemize}
\end{for_you}

\subsection{Confidentialité}

Dans cette sous-section, vous devez détailler les mesures de protection des données et le respect de la vie privée. Le jury attend une approche conforme au RGPD et aux bonnes pratiques de sécurité.

\begin{for_you}{Mesures de confidentialité :}
    
    \begin{itemize}
        \item \textbf{Chiffrement :} Données sensibles chiffrées en base
        \item \textbf{Securiter :} Suppression de badge auto.
        \item \textbf{Accès contrôlé :} Logs d'accès
        \item \textbf{RGPD :} Consentement, droit à l'oubli, portabilité
        \item \textbf{Anonymisation :} Données anonymisées pour les rapports
    \end{itemize}
\end{for_you}

\subsection{Droits d'accès}

Dans cette sous-section, vous devez présenter votre système de permissions et de contrôle d'accès. Le jury attend une matrice claire des droits par rôle et fonctionnalité.

\begin{for_you}{Matrice de droits :}

    \begin{center}
        \begin{tabular}{|l|l|l|l|l|}
            \hline
            \textbf{Rôle} & \textbf{App} & \textbf{Badge} \\
            \hline
            Admin & \mycheckmark & \mycheckmark \\
            Coatchs & \mycross & \mycheckmark \\
            Etudiants & \mycross & \mycheckmark \\
            Inviter & \mycross & \mycheckmark \\
            \hline
        \end{tabular}
    \end{center}
\end{for_you}

\subsection{Authentification}

Dans cette sous-section, vous devez détailler votre système d'authentification et de gestion des sessions. Le jury attend une approche sécurisée et robuste.

\begin{for_you}{Système d'authentification :}
    \textbf{système d'authentification :}
    \begin{itemize}
        \item \textbf{Connexion :} Identifient/Mot de passe(JWT), face id...
        \item \textbf{Sécurité :} Mot de passe fort, Stockage du badge en local
        \item \textbf{Récupération :} Par email sécurisé
    \end{itemize}
    J'ai decider de choisir JWT car c'est le seul moyen d'autentification que j'ai pue utiliser. Ajoutez face ID comme moyen de connexion est aussi interessant car il permet si nous posedons 
    un appareil apple d'augmenter la securiter et la fluidifer de la conextion.
\end{for_you}

\begin{conseil}
    \begin{itemize}
        \item Spécifier toutes les fonctionnalités front-end et back-end
        \item Définir clairement les rôles et droits d'accès
        \item Documenter les exigences de confidentialité
        \item Prévoir les mécanismes d'authentification et d'autorisation
        \item Valider les exigences avec les utilisateurs métier
    \end{itemize}
\end{conseil}

\begin{jury}
    \begin{itemize}
        \item Quelles sont vos exigences fonctionnelles prioritaires ?
        \item Comment gérez-vous les droits d'accès ?
        \item Vos exigences sont-elles testables ?
        \item Avez-vous prévu la confidentialité des données ?
        \item Comment validez-vous les exigences avec les utilisateurs ?
    \end{itemize}
\end{jury}

\section{Exigences et choix techniques}

L'architecture 3 tiers (présentation, logique métier, données) offre une séparation claire des responsabilités et facilite la maintenance. PostgreSQL assure la cohérence transactionnelle des données métier, tandis que MongoDB optimise le stockage des rapports et logs grâce à sa flexibilité documentaire. Cette approche hybride maximise les performances selon le type de données traitées.

Les choix techniques sont guidés par les exigences de performance, de scalabilité et de maintenabilité. L'utilisation de technologies éprouvées réduit les risques techniques tout en permettant une évolution progressive de la solution.

\subsection{Exigences}

Dans cette sous-section, vous devez détailler toutes les contraintes techniques et les exigences non-fonctionnelles de votre système. Le jury attend une analyse complète des performances, sécurité, scalabilité et maintenabilité.

\begin{for_you}{Exigences techniques :}

    \begin{itemize}
        \item \textbf{Performance :} Temps de réponse < 5s.
        \item \textbf{Sécurité :} Chiffrement obligatoire, Digicode a partir d'une certaine heur en complement du badge, log.
        \item \textbf{Multi Plateforme :} Soit disponible sur Apple et Android mais egalement pour des potenitel badge physique.
        \item \textbf{Maintenabilité :} Code documenté, tests automatisés
    \end{itemize}
\end{for_you}

\subsection{Choix}

Dans cette sous-section, vous devez justifier vos choix technologiques par rapport aux exigences identifiées. Le jury attend une analyse comparative et une justification claire de chaque décision technique.

\textbf{Vos choix techniques :} \textit{\newline
    J'ai choisi Elixir car mon systeme est sencer etre fusionner avec l'intra de l'ecole qui est egalement coder en elixir donc nous gardons une coerence logiciel meme chose pour la data base.
    En ce qui concerne Phoenix on pourrais le comparer a react pour js mais lui est le framework d'elixir il est donc parfait pour cette utilisation nous pouvons utiliser avec cela live vue qui est un outil pour faire du front}
\newline

\begin{for_you}
    \textbf{Architecture logique simplifiée :}
    \begin{verbatim}
    +-------------------+    +-------------------+    +-------------------+
    |   Frontend        |    |   Backend         |    |   Databases       |
    |   (Phoenix,       |<-->|   (Elixir)        |<-->|   PostgreSQL      |
    |    Live vue)      |    |                   |    |                   |
    +-------------------+    +-------------------+    +-------------------+
    \end{verbatim}

    \textbf{Exemple de modèle PostgreSQL :}
    \begin{lstlisting}[language=SQL]
    CREATE TABLE users (
        id SERIAL PRIMARY KEY,
        email VARCHAR(255) UNIQUE NOT NULL,
        role_id INTEGER REFERENCES roles(id),
        int badge
    );

    CREATE INDEX idx_users_email ON users(email);
    \end{lstlisting}
\end{for_you}

\begin{conseil}
\begin{itemize}
    \item Justifier chaque choix technique par des critères objectifs
    \item Documenter l'architecture 3 tiers et les responsabilités
    \item Expliquer l'utilisation de PostgreSQL et MongoDB
    \item Prévoir l'évolution et la scalabilité de l'architecture
    \item Évaluer les alternatives techniques considérées
\end{itemize}
\end{conseil}

\begin{jury}
\begin{itemize}
    \item Pourquoi avez-vous choisi cette architecture ?
    \item Comment justifiez-vous l'utilisation de deux bases de données ?
    \item Votre architecture est-elle scalable ?
    \item Quels sont les points de défaillance potentiels ?
    \item Avez-vous considéré des alternatives techniques ?
\end{itemize}
\end{jury}

\section{Définition du MVP}

Le MVP concentre les fonctionnalités essentielles pour valider l'hypothèse produit : authentification, gestion des projets de base, et tableau de bord simple. Cette approche permet d'obtenir un retour utilisateur précoce et d'ajuster la roadmap en conséquence. Les scénarios essentiels couvrent les cas d'usage les plus fréquents et critiques pour le métier.

\textbf{Votre définition du V1 :} \textit{Mon project doit être aux minimum composer d'une application. Sont rôle sera de gerer les badge des user
tout en gardent un trace du passage des gens dans le batiment.
\newline}

\begin{for_you}
    \textbf{Scénarios essentiels V1 :}
    \begin{enumerate}
        \item Connexion utilisateur et gestion de session
        \item Création et modification d'un badge
        \item Donée sur les entrées/sories
        \item Chiffrement des données
        \item Génération de rapports basiques
    \end{enumerate}
\end{for_you}

\begin{conseil}
    \begin{itemize}
        \item Délimiter précisément le périmètre du MVP
        \item Identifier les scénarios essentiels prioritaires
        \item Valider chaque fonctionnalité avec les utilisateurs
        \item Mesurer l'impact de chaque feature
        \item Prévoir des critères de succès clairs
    \end{itemize}
\end{conseil}

\begin{jury}
    \begin{itemize}
        \item Votre MVP est-il vraiment minimal ?
        \item Quels sont vos scénarios essentiels ?
        \item Comment validez-vous chaque fonctionnalité ?
        \item Avez-vous des critères de succès mesurables ?
        \item Comment gérez-vous les demandes hors périmètre ?
    \end{itemize}
\end{jury}

\section{Roadmap}

La roadmap v1 vers v2 prévoit l'ajout progressif de fonctionnalités avancées basées sur les retours utilisateurs et les besoins métier émergents. Cette approche itérative minimise les risques et optimise l'allocation des ressources.

\textbf{Votre roadmap :}

\begin{for_you}
\textbf{Roadmap v1 \myarrow v2 :}
    \begin{itemize}
        \item \textbf{v1.0 :} Fonctionnalités de base (MVP)
        \item \textbf{v1.1 :} Compatibilitée telephone
        \item \textbf{v1.2 :} Intégrations externes (API)
        \item \textbf{v2.0 :} crosse Plateforme Apple/Android
    \end{itemize}
\end{for_you}

\begin{conseil}
\begin{itemize}
    \item Délimiter précisément le périmètre du MVP
    \item Identifier les scénarios essentiels prioritaires
    \item Planifier la roadmap v1 vers v2 de manière réaliste
    \item Prévoir des jalons de validation utilisateur
    \item Documenter les critères de passage de version
\end{itemize}
\end{conseil}

\begin{jury}
\begin{itemize}
    \item Votre MVP est-il vraiment minimal ?
    \item Quels sont vos scénarios essentiels ?
    \item Comment validez-vous le passage en v2 ?
    \item Votre roadmap est-elle réaliste ?
    \item Comment gérez-vous les changements de priorité ?
\end{itemize}
\end{jury}

\section{Liens utiles}

\begin{itemize}
    \item User Stories: \url{https://www.mountaingoatsoftware.com/agile/user-stories}
    \item MoSCoW: \url{https://www.productplan.com/glossary/moscow-prioritization/}
    \item PostgreSQL Docs: \url{https://www.postgresql.org/docs/}
    \item MongoDB Modeling: \url{https://bit.ly/mongodb-modeling}
    \item Architecture 3-tier: \url{https://en.wikipedia.org/wiki/Multitier_architecture}
\end{itemize}
