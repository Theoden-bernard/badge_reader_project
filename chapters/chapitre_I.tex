\chapter{Présentation personnelle et du projet}

\section{Rôle du candidat et contexte}

Dans cette section, vous devez présenter votre rôle dans le projet et le contexte organisationnel. Le jury attend une explication claire de vos responsabilités et de l'environnement de travail dans lequel s'inscrit votre projet CDA.

\textbf{Votre rôle :} \textit{[Décrivez votre fonction exacte : développeur full-stack, frontend, backend, DevOps, etc. Précisez vos responsabilités principales et votre niveau d'autonomie]}

\textbf{Contexte organisationnel :} \textit{[Expliquez l'entreprise, l'équipe, les enjeux métier, et la problématique que votre projet doit résoudre. Mentionnez les contraintes techniques et organisationnelles]}

\textbf{Durée et planning :} \textit{[Précisez la durée de votre alternance, les phases du projet, et les jalons importants]}

\begin{exemple}
\textbf{Exemple de pitch QQOQCP :}
\begin{itemize}
    \item \textbf{Quoi :} Application de gestion des projets internes
    \item \textbf{Qui :} Équipes de développement et management
    \item \textbf{Où :} Environnement cloud hybride
    \item \textbf{Quand :} Déploiement progressif sur 6 mois
    \item \textbf{Comment :} Architecture microservices avec React/Node.js
    \item \textbf{Pourquoi :} Centraliser et optimiser le suivi des projets
\end{itemize}

\textbf{Votre pitch QQOQCP :} \textit{[Rédigez votre propre pitch en suivant cette structure]}
\end{exemple}

\begin{conseil}
\textbf{Ce que le jury attend dans cette section :}
\begin{itemize}
    \item Une présentation claire de votre rôle et de vos responsabilités
    \item Une explication du contexte métier et des enjeux
    \item Une justification de la pertinence du projet
    \item Un pitch QQOQCP synthétique et percutant
    \item Des indicateurs de succès mesurables
\end{itemize}

\textbf{Conseils de rédaction :}
\begin{itemize}
    \item Soyez précis sur votre fonction (évitez les généralités)
    \item Montrez votre compréhension des enjeux métier
    \item Justifiez le choix de votre projet
    \item Utilisez des chiffres et des métriques quand c'est possible
\end{itemize}
\end{conseil}

\begin{jury}
\begin{itemize}
    \item Pouvez-vous présenter votre rôle précis dans ce projet ?
    \item Quel est le contexte métier de votre entreprise ?
    \item Quels sont les enjeux techniques principaux ?
    \item Comment mesurez-vous le succès de votre projet ?
    \item Quelles sont les contraintes temporelles et budgétaires ?
\end{itemize}
\end{jury}

\section{Problématique et objectifs SMART}

Dans cette section, vous devez identifier clairement la problématique que votre projet résout et définir des objectifs SMART mesurables. Le jury attend une analyse précise des enjeux et des bénéfices attendus.

\textbf{Problématique identifiée :} \textit{[Décrivez le problème métier ou technique que votre projet résout. Soyez spécifique sur les impacts négatifs actuels]}

\textbf{Objectifs SMART :} \textit{[Définissez 3 à 5 objectifs spécifiques, mesurables, atteignables, pertinents et temporellement définis]}

\textbf{Bénéfices attendus :} \textit{[Listez les améliorations concrètes que votre projet apportera]}

\begin{exemple}
\textbf{Exemple d'objectifs SMART :}
\begin{itemize}
    \item \textbf{Spécifique :} Développer une plateforme unifiée de gestion de projet
    \item \textbf{Mesurable :} Réduire de 40\% le temps de reporting hebdomadaire
    \item \textbf{Atteignable :} Livrer la v1 en 6 mois avec l'équipe actuelle
    \item \textbf{Pertinent :} Aligner les outils sur la stratégie digitale
    \item \textbf{Temporel :} Déploiement complet avant fin d'année 2025
\end{itemize}

\textbf{Vos objectifs SMART :} \textit{[Rédigez vos propres objectifs en suivant cette structure]}

\textbf{Exemple de diagramme de contexte :}
\begin{verbatim}
                    +=================================+
                    |        VOTRE PROJET CDA         |
                    |                                 |
                    +=================================+
                                      |
                    +-----------------+-----------------+
                    |                 |                 |
            +-------+-------+   +-------+-------+   +-------+-------+
            | Équipe       |   | Direction     |   | Utilisateurs |
            | technique    |   | métier        |   | finaux       |
            +---------------+   +---------------+   +---------------+
                    |                 |                 |
                    +-----------------+-----------------+
                                      |
                              +-------+-------+
                              | Systèmes      |
                              | externes      |
                              +---------------+
\end{verbatim}

\textbf{Votre diagramme de contexte :} \textit{[Créez un diagramme montrant les acteurs et leurs interactions avec votre projet]}
\end{exemple}

\begin{conseil}
\textbf{Ce que le jury attend dans cette section :}
\begin{itemize}
    \item Une problématique clairement identifiée et justifiée
    \item Des objectifs SMART précis et mesurables
    \item Une compréhension des enjeux métier
    \item Des indicateurs de succès quantifiés
    \item Un diagramme de contexte montrant les acteurs
\end{itemize}

\textbf{Conseils de rédaction :}
\begin{itemize}
    \item Soyez spécifique sur les impacts négatifs actuels
    \item Quantifiez vos objectifs (pourcentages, délais, volumes)
    \item Montrez la pertinence métier de votre projet
    \item Utilisez des diagrammes pour clarifier les interactions
\end{itemize}
\end{conseil}

\begin{jury}
\textbf{Questions de contrôle du jury :}
\begin{itemize}
    \item Pouvez-vous expliquer clairement la problématique que votre projet résout ?
    \item Vos objectifs sont-ils vraiment SMART (spécifiques, mesurables, atteignables, pertinents, temporels) ?
    \item Comment mesurez-vous le succès de votre projet ?
    \item Quels sont les bénéfices attendus pour l'entreprise ?
    \item Pouvez-vous présenter un diagramme de contexte de votre projet ?
    \item Quelles sont les contraintes temporelles et budgétaires ?
\end{itemize}
\end{jury}

\section{Liens utiles}

\begin{itemize}
    \item GitHub About: \url{https://docs.github.com/}
    \item SMART Goals: \url{https://bit.ly/smart-goals-atlassian}
    \item Project Management Institute: \url{https://www.pmi.org/}
    \item Agile Manifesto: \url{https://agilemanifesto.org/}
    \item Business Model Canvas: \url{https://bit.ly/business-model-canvas}
\end{itemize}
