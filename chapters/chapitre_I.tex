\chapter{Présentation personnelle et du projet}

\section{Rôle du candidat et contexte}

Dans cette section, vous devez présenter votre rôle dans le projet et le contexte organisationnel. Le jury attend une explication claire de vos responsabilités et de l'environnement de travail dans lequel s'inscrit votre projet CDA.

\textbf{Votre rôle :} \textit{[Décrivez votre fonction exacte : développeur full-stack, frontend, backend, DevOps, etc. Précisez vos responsabilités principales et votre niveau d'autonomie]}

\textbf{Contexte organisationnel :} \textit{[Expliquez l'entreprise, l'équipe, les enjeux métier, et la problématique que votre projet doit résoudre. Mentionnez les contraintes techniques et organisationnelles]}

\textbf{Durée et planning :} \textit{[Précisez la durée de votre alternance, les phases du projet, et les jalons importants]}\newline

\begin{exemple}
\textbf{Exemple de pitch QQOQCP :}
\begin{itemize}
    \item \textbf{Quoi :} Application de gestion des projets internes
    \item \textbf{Qui :} Équipes de développement et management
    \item \textbf{Où :} Environnement cloud hybride
    \item \textbf{Quand :} Déploiement progressif sur 6 mois
    \item \textbf{Comment :} Architecture microservices avec React/Node.js
    \item \textbf{Pourquoi :} Centraliser et optimiser le suivi des projets 
    \newline
\end{itemize}

\textbf{Votre pitch QQOQCP :} \textit{Je vous présent un système de badge qui sera dédier aux membre de colint mais également aux inviter. 
J'ai imaginer un système dans le quelles chaque étudiant est assigner a un badge qui lui permet de rentrée mais également de sortir de l'ecole.
Mais egalement un systeme de generation temporaire de badge pour les inviter de colint comme par exemple des parents ou future etudiant interesser par l'ecole.
\newline
Tout ça combinée a un systeme de \underline{rôle} qui donnera des accès diffèrent. Par exemple a parir de 18h les etudiants ne peuve plus rantre dans l'école toute sortie est donc définitive ou il devra faire appel a un membre du staff pour rentrée car lui peux rentrée jusqu'a 20h.
Le deploiment de ce systeme devra être mis en place le plus taud paussible mais je n'ai pas encore de date exacte a vous donnée aujourd'hui par manque d'expertise a ce niveau la.
\newline
\newline
Toute une application de statistiquessur sur les gens qui rentre et qui sorte du batiment mais egalement la gestion de badge sera developper en \underline{Elixir} avec sont framework \underline{Phoenix}}. Étant donée que ce systeme de badge sera relier a l'intra de l'ecole j'ai donc choisi Elixir pour rester dans un environnement cohérent.
\end{exemple}

\begin{conseil}
    \textbf{Ce que le jury attend dans cette section :}
    \begin{itemize}
        \item Une présentation claire de votre rôle et de vos responsabilités
        \item Une explication du contexte métier et des enjeux
        \item Une justification de la pertinence du projet
        \item Un pitch QQOQCP synthétique et percutant
        \item Des indicateurs de succès mesurables
    \end{itemize}

    \textbf{Conseils de rédaction :}
    \begin{itemize}
        \item Soyez précis sur votre fonction (évitez les généralités)
        \item Montrez votre compréhension des enjeux métier
        \item Justifiez le choix de votre projet
        \item Utilisez des chiffres et des métriques quand c'est possible
    \end{itemize}
\end{conseil}

\begin{jury}
    \begin{itemize}
        \item Pouvez-vous présenter votre rôle précis dans ce projet ?
        \item Quel est le contexte métier de votre entreprise ?
        \item Quels sont les enjeux techniques principaux ?
        \item Comment mesurez-vous le succès de votre projet ?
        \item Quelles sont les contraintes temporelles et budgétaires ?
    \end{itemize}
\end{jury}

\section{Problématique et objectifs SMART}

Dans cette section, vous devez identifier clairement la problématique que votre projet résout et définir des objectifs SMART mesurables. Le jury attend une analyse précise des enjeux et des bénéfices attendus.

\textbf{Problématique identifiée :} \textit{\newline 
Colint school est une ecole dit libre. Je m'explique. Dans cette ecole on nous donne la possibilitée d'avancer a notre ritme exemple si on me donne un project et que j'ai besoin de 2 semaine pour le rendre c'est okay.
si j'ai besoin de moins c'est egalement okay et donc je peuc passer a la suite. Mais également dans cette école. Nous somme libre de venir a l'ecole et de partir globalement quand nous le souhaiton. Mais nous somme malheureusement contraintes
par des problématique d'ouvertur/fermetur d'école. Car il faudrait encormement de clef pour tout les etudiants rajouter a ça que nous pouvons pas s'avoir qui rentre qui sort et vous obtener des probleme de securitée aux sein de l'ecole.
}

\textbf{Objectifs SMART :} \textit{\newline
    J'aimerait donc dans un premier temps crée une application de stat qui me permetterait par la suit de s'avoir qui rentre/sort de l'ecole tout en pouvant gerer les badge et les rôle qui donnerait des acces different en fonction de problematique. 
    Suite à ça j'aimerait integrer un systeme ou chaque personne reçois un mail ou a l'interieur il aurait un badge numerique qui utiliserait l'NFC de leur Telephone. Et pour finir ajouter ceci a un wallet sur le Telephone.
    Pour que cela fonction sur Iphone/Android.}

\textbf{Bénéfices attendus :} \textit{\newline
    Tout ceci permetterait de fludifier le trafique aux seins de Colint mais egalement de permettre aux etudiant qui souhait travailler plus tard de ne plus etre limiter par cette problematique tout en gardent 
    une trace des induvidu qui rentre/sorte et dons de securiser le batiment.}

\begin{exemple}
    \textbf{Exemple d'objectifs SMART :}
    \begin{itemize}
        \item \textbf{Spécifique :} Développer une plateforme unifiée de gestion de projet
        \item \textbf{Mesurable :} Réduire de 40\% le temps de reporting hebdomadaire
        \item \textbf{Atteignable :} Livrer la v1 en 6 mois avec l'équipe actuelle
        \item \textbf{Pertinent :} Aligner les outils sur la stratégie digitale
        \item \textbf{Temporel :} Déploiement complet avant fin d'année 2025
    \end{itemize}

    \textbf{Vos objectifs SMART :} \textit{\newline
    1. Decouper le project en plusieur étape realisable et logique. \newline
    3. Comprendre des techno encore incounus ou non metriser comme l'UML, les test unitaire et autre. \newline
    2. Fase de conception. \newline
    4. Comprendre le NFC et les norme autours de ce dernier.\newline
    5. Création de l'application.
    }

    \textbf{Votre diagramme de contexte :}

    \begin{center}
        \includegraphics[width=15cm,height=80mm]{./assets/diagramme_de_context.png}
    \end{center}
\end{exemple}

\begin{conseil}
    \textbf{Ce que le jury attend dans cette section :}
    \begin{itemize}
        \item Une problématique clairement identifiée et justifiée
        \item Des objectifs SMART précis et mesurables
        \item Une compréhension des enjeux métier
        \item Des indicateurs de succès quantifiés
        \item Un diagramme de contexte montrant les acteurs
    \end{itemize}

    \textbf{Conseils de rédaction :}
    \begin{itemize}
        \item Soyez spécifique sur les impacts négatifs actuels
        \item Quantifiez vos objectifs (pourcentages, délais, volumes)
        \item Montrez la pertinence métier de votre projet
        \item Utilisez des diagrammes pour clarifier les interactions
    \end{itemize}
\end{conseil}

\begin{jury}
    \textbf{Questions de contrôle du jury :}
    \begin{itemize}
        \item Pouvez-vous expliquer clairement la problématique que votre projet résout ?
        \item Vos objectifs sont-ils vraiment SMART (spécifiques, mesurables, atteignables, pertinents, temporels) ?
        \item Comment mesurez-vous le succès de votre projet ?
        \item Quels sont les bénéfices attendus pour l'entreprise ?
        \item Pouvez-vous présenter un diagramme de contexte de votre projet ?
        \item Quelles sont les contraintes temporelles et budgétaires ?
    \end{itemize}
\end{jury}

\section{Liens utiles}

\begin{itemize}
    \item GitHub About: \url{https://docs.github.com/}
    \item SMART Goals: \url{https://bit.ly/smart-goals-atlassian}
    \item Project Management Institute: \url{https://www.pmi.org/}
    \item Agile Manifesto: \url{https://agilemanifesto.org/}
    \item Business Model Canvas: \url{https://bit.ly/business-model-canvas}
\end{itemize}
