\chapter{Présentation personnelle et du projet}

\section{Rôle du candidat et contexte}

Dans cette section, vous devez présenter votre rôle dans le projet et le contexte organisationnel. Le jury attend une explication claire de vos responsabilités et de l'environnement de travail dans lequel s'inscrit votre projet CDA.

\textbf{Votre rôle :} \textit{Dans le cadre de mon PFE j'aimerais developper une application et gere des badge. je vais donc devoir conceptioner et developer cette derniere.}

\textbf{Contexte organisationnel :} \textit{J'utiliserait GitHub project avec un systeme de Kanban pour s'organiser.}

\textbf{Durée et planning :} \textit{Je suis en alternence du lundi aux mercredi et en cours le jeudi et le vendredi}\newline

\begin{for_you}{pitch QQOQCP :}
\begin{itemize}
    \item \textbf{Quoi :} Application de gestion des projets internes
    \item \textbf{Qui :} Équipes de développement et management
    \item \textbf{Où :} Environnement cloud hybride
    \item \textbf{Quand :} Déploiement progressif sur 6 mois
    \item \textbf{Comment :} Architecture microservices avec React/Node.js
    \item \textbf{Pourquoi :} Centraliser et optimiser le suivi des projets 
    \newline
\end{itemize}

\textbf{Votre pitch QQOQCP :} \textit{Je vous présent un système de badge qui sera dédier aux membre de colint mais également aux inviter. 
J'ai imaginer un système dans le quelles chaque étudiant est assigner a un badge qui lui permet de rentrée mais également de sortir de l'ecole.
Mais egalement un systeme de generation temporaire de badge pour les inviter de colint comme par exemple des parents ou future etudiant interesser par l'ecole.
\newline
Tout ça combinée a un systeme de \underline{rôle} qui donnera des accès diffèrent. Par exemple a parir de 18h les etudiants ne peuve plus rantre dans l'école toute sortie est donc définitive ou il devra faire appel a un membre du staff pour rentrée car lui peux rentrée jusqu'a 20h.
Le deploiment de ce systeme devra être mis en place le plus taud paussible mais je n'ai pas encore de date exacte a vous donnée aujourd'hui par manque d'expertise a ce niveau la.
\newline
\newline
Toute une application de controle d'accessur pour les gens qui rentre et qui sorte du batiment mais egalement la gestion de badge sera developper en \underline{Elixir} avec sont framework \underline{Phoenix}}. Étant donée que ce systeme de badge sera relier a l'intra de l'ecole j'ai donc choisi Elixir pour rester dans un environnement cohérent.
\end{for_you}

\begin{conseil}
    \textbf{Ce que le jury attend dans cette section :}
    \begin{itemize}
        \item Une présentation claire de votre rôle et de vos responsabilités
        \item Une explication du contexte métier et des enjeux
        \item Une justification de la pertinence du projet
        \item Un pitch QQOQCP synthétique et percutant
        \item Des indicateurs de succès mesurables
    \end{itemize}

    \textbf{Conseils de rédaction :}
    \begin{itemize}
        \item Soyez précis sur votre fonction (évitez les généralités)
        \item Montrez votre compréhension des enjeux métier
        \item Justifiez le choix de votre projet
        \item Utilisez des chiffres et des métriques quand c'est possible
    \end{itemize}
\end{conseil}

\begin{jury}
    \begin{itemize}
        \item Pouvez-vous présenter votre rôle précis dans ce projet ?
        \item Quel est le contexte métier de votre entreprise ?
        \item Quels sont les enjeux techniques principaux ?
        \item Comment mesurez-vous le succès de votre projet ?
        \item Quelles sont les contraintes temporelles et budgétaires ?
    \end{itemize}
\end{jury}

\section{Problématique et objectifs SMART}

Dans cette section, vous devez identifier clairement la problématique que votre projet résout et définir des objectifs SMART mesurables. Le jury attend une analyse précise des enjeux et des bénéfices attendus.
\newline

\begin{for_you}{Problématique identifiée :}
    \textit{
        \newline 
        Comment peut on fluidifier, contrôler et sécuriser l’accès aux bâtiments de Colint pour permettre aux étudiants et membre du staff de pénétrer dans l’école ?
        \newline 
    }
    
    \textbf{Objectifs SMART :} \textit{
        \newline
        \underline{Dans un premier temps j'aimerais} : Crée une application qui récuppere des statistiques sur les utilisateur qui Rentre/Sorte de l'école. Cette même aplication pourra gerer les badges et rôles des utilisateurs
        et pour finir, crée des badge. Il serait interessant de livrées une V1 dans 3 mois apres décembre. il est important que l'application soit simple et miimaliste.
        \newline
        \newline 
        \underline{Dans un secondes temps j'aimerais} : Gere un systeme de badge directement sur un telephone, qu'il soit Iphone/Android cela devrait etre fait dans les 3 mois suivant apres l'application principale.
        \newline
        \newline
        Le tout devrai etre livrée pour fin 2026
        \newline 
        }
    
    \textbf{Bénéfices attendus :} \textit{
        \newline
        Fludifier et Securiser le trafique aux sein du batiment. S'assurer la legitimiter des personne qui rentre dans le batiment.
        \newline 
        }
\end{for_you}

\begin{for_you}{Objectifs S.M.A.R.T et diagramme de contexte :}
    
    \textbf{Objectifs S.M.A.R.T :}
    \begin{itemize}
        \item \textbf{Spécifique :} Développer une application de contrloe d'acces
        \item \textbf{Mesurable :} Réduire de 40\% le temps de reporting hebdomadaire
        \item \textbf{Atteignable :} Livrer la v1 en 6 mois a partir de decembre
        \item \textbf{Pertinent :} Application symple d'utilisation et intuitive
        \item \textbf{Temporel :} Déploiement complet avant fin d'année 2026
        \newline
    \end{itemize}

    \textbf{Diagramme de contexte :}
    \begin{center}
        \includegraphics[width=15cm,height=80mm]{./assets/uml/diagramme_de_context.png}
    \end{center}

\end{for_you}

\begin{conseil}
    \textbf{Ce que le jury attend dans cette section :}
    \begin{itemize}
        \item Une problématique clairement identifiée et justifiée
        \item Des objectifs SMART précis et mesurables
        \item Une compréhension des enjeux métier
        \item Des indicateurs de succès quantifiés
        \item Un diagramme de contexte montrant les acteurs
    \end{itemize}

    \textbf{Conseils de rédaction :}
    \begin{itemize}
        \item Soyez spécifique sur les impacts négatifs actuels
        \item Quantifiez vos objectifs (pourcentages, délais, volumes)
        \item Montrez la pertinence métier de votre projet
        \item Utilisez des diagrammes pour clarifier les interactions
    \end{itemize}
\end{conseil}

\begin{jury}
    \textbf{Questions de contrôle du jury :}
    \begin{itemize}
        \item Pouvez-vous expliquer clairement la problématique que votre projet résout ?
        \item Vos objectifs sont-ils vraiment SMART (spécifiques, mesurables, atteignables, pertinents, temporels) ?
        \item Comment mesurez-vous le succès de votre projet ?
        \item Quels sont les bénéfices attendus pour l'entreprise ?
        \item Pouvez-vous présenter un diagramme de contexte de votre projet ?
        \item Quelles sont les contraintes temporelles et budgétaires ?
    \end{itemize}
\end{jury}

\section{Liens utiles}

\begin{itemize}
    \item GitHub About: \url{https://docs.github.com/}
    \item SMART Goals: \url{https://bit.ly/smart-goals-atlassian}
    \item Project Management Institute: \url{https://www.pmi.org/}
    \item Agile Manifesto: \url{https://agilemanifesto.org/}
    \item Business Model Canvas: \url{https://bit.ly/business-model-canvas}
\end{itemize}
